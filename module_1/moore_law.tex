\documentclass{article}

\usepackage{csquotes}
\usepackage{booktabs}

\author{Martín Steven Hernández Ortiz}
\title{Analysing the Moore's Law in the Current Age}

\newcommand{\micm}{\micro\meter}

\begin{document}
\maketitle

In order the analyse the Moore's Law, we need to define that it states and see 
if it's really a Law of Science, ones which describe or predict the behaviour of 
natural phenomena.

\blockquote[Gordon Moore. 1965.]{The number of transistors in an integrated circuit (IC) doubles about every two years}

From starters, this law is not a proper one but an observation.
Mainly due to the fact that in 1965 there wasn't that much of advancements in 
computacional hardware and the process of designing, making and having more 
transistor-dense processors. So, in that time and age, Moore couldn't have 
stated that law in a realistic way.

Now, in the modern age of the computer, specially on the end of the 2000s we 
had a major advancements, this observation saw an plateu of the size of transistros 
and hence density of transistors per processor. We can see the data since the 1999 to 2022:

\begin{tabular}{lc}
    \toprule
    Transistor Size & Year \\
    \midrule
    180 nm & 1999 \\
    130 nm & 2001 \\
    90 nm & 2003 \\
    65 nm & 2005 \\
    45 nm & 2007 \\
    32 nm & 2009 \\
    28 nm & 2010 \\
    22 nm & 2012 \\
    14 nm & 2014 \\
    10 nm & 2016 \\
    7 nm & 2018 \\
    5 nm & 2020 \\
    3 nm & 2022 \\
    \bottomrule
\end{tabular}

This plateu of the size of the transistors are mainly due to the trouble of having to handle the head generated 
by the consumption of energy by the transistors in order to avoid the structure and processor to melt and destroy itself.

The energy consumption is dictaded by dynamic power in the system, defined by the ecuation:
\[
    P = \alpha \cdot CFV^{2}
\]
Where, the main part in the matter here is $V$ which is the voltage of the system.
Which needs to scaled with the size of the transistors, in a direct relation. Meaning that, the lower $V$ is 
the smaller the transistors can be. But there's a problemn with this, since when we have very low $V$ from small 
transistors, there could be some problemns while measuring the $V$ in the system, mainly due to noice and leakage.


Meaning that if our $V$ is extreamly small, it can't have a lot of thershold to check if the information it has is correct.
In order to avoid this, it should have a bigger input of $V$, but as described earlier, the more $V$ and power consumption 
is required, the transistors will generate more heat which in smaller scales could damage or destroy them.


Finally, since discovering, fixing and making better and smaller transistors has become way more harder since 1965, and having a 
plateu of the size of transistors in the 2000s, the Moore observation is incorrect since the modern day of the computer.

\end{document}
